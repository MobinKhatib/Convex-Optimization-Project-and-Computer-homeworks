

%\subsection*{\texttt{center} environment: (standard \LaTeX{})}
\begin{titlepage}
\begin{center}

\Huge
\textbf{به نام خدا}\\[2cm]

\huge
\textbf{تمرین کامپیوتری دوم بهینه سازی محدب }\\[0.5cm]
\textbf{دکتر یاسایی}\\[1.5cm]

\Large
\textbf{مبین خطیب}\\[0.3cm]
\textbf{99106114}\\[0.5cm]







\begin{center}
    \includegraphics[width=0.5\textwidth]{logo.jpg}
\end{center}
\end{center}
\end{titlepage}




\newpage
\huge
\section{ برآورد درست نمایی بیشینه}
\large

1.1:

\lr{log-likelihood}
برای مشاهده $ N_t $ رخداد برابر است با :

$ -\lambda_t - N_t\log\lambda_t + \log N_t! $

 از آنجایی که این رخداد ها به دلیل توزیع پواسون مستقل هستند میتوانیم همه این 24 رخداد را با هم در نظر بگیریم که بتوانیم به محاسبه  \lr{log-likelihood} نزدیک تر شویم بنابراین \lr{negative log-likelihood} برای مجموعه مشاهدات رویداد ها برابر است با :
\begin{equation}
\Sigma^{24}_{t=1} = \lambda_t - N_t \log\lambda_t + \log(N_t!)
\end{equation}
بنابراین تخمین بیشینه درست نمایی برای $ \lambda_t $  مینیمم خواهد کرد \lr{negative log-likelihood} روی مشاهدات $ N_1, .....,N_24 $  که این مینیمایز کردن با حل مسئله بهینه سازی زیر برای $ \lambda_1,...,\lambda_24 $ انجام خواهد شد:
\begin{equation}
\begin{aligned}
\text{minimize} \ \Sigma^{24}_{t=1} = \lambda_t - N_t \log\lambda_t + \log(N_t!)\\
 \text{to subject} \ \lambda \geq 0 \\
\end{aligned}
\end{equation}

اما در پاسخ بالا $ N_t = 0 $ کاور نمیشود چون فقط $ \lambda_t > 0 $ در نظر گرفته ایم برای این حالت میتوانیم مینیمایزشن را روی هر $ \lambda_t $ انجام دهیم اول فرض میکنیم $ N_t > 0 $ باشد آنگاه مینیمایزر $ 1 - N_t/\lambda_t = 0  $ را ارضا خواهد کرد بنابراین $ \lambda_t = N_t $ برای $ N_t = 0 $ خواهد بود در واقع ما $ \lambda_t $ را به قید $ \lambda_t \ge 0 $ مینیمایز کردیم که در نتیجه این خواهیم داشت $ \lambda_t = 0 $ بنابراین در همه حالت ها تخمین ML ما خواهد بود $ N_t = \lambda_t $ در واقع اگر مشاهدات ما $ N_t $ رخداد باشد احتمالا میانگین این اتفاق ها رخ داده است.

2.1.

برای این کار باید مسئله بهینه سازی زیر را حل کنیم:

\begin{equation}
\begin{aligned}
\text{minimize} \ \left( \sum_{t=1}^{24} ( \lambda_t - N_t \log \lambda_t ) + \sum_{t=1}^{23} (\lambda_{t+1} - \lambda_t)^2 + (\lambda_1 - \lambda_{24})^2 \right)\\
\textrm{to subject} \ \lambda \geq 0 
\end{aligned}
\end{equation}

 به طوری که $\lambda \in R^{24} $ ترم ثابت را در نظر نگرفتیم چون وابسته به $ \lambda_t $ نیست



3.1:
هرچه 
$ \rho \rightarrow \infty $ 
میل دهیم $ \lambda_t $ ها به برابری نزدیک تر میشوند بنابراین وقتی داشته باشیم 
$\lambda = \tilde{\lambda}\mathbf{1} $
بنابراین خواهیم داشت:
\begin{equation}
\text{minimize}\quad 24\tilde{\lambda} - \left(\sum_{t=1}^{24}N_t \log(\tilde{\lambda})\right)\\
\quad \text{to subject} \quad \tilde{\lambda} \geq 0
\end{equation}
در واقع راه حل ما مدل پواسون در حالت ثابت است که در آن
$ \lambda_t = \tilde{\lambda} = \bar{N} = \Sigma_i N_i/24 $ 
که این میانگین رخداد های هر ساعت از روز که در کل روز رخ داده میباشد.

5,4.1:

 \begin{center}
    \includegraphics[width=0.9\textwidth]{1.4&5.jpg}
\end{center}

\newpage
\huge
\section{سطوح فعالیت بهینه}
\large

1.2:

مسئله بهینه سازی را میتوانیم به صورت زیر عنوان کنیم:

\begin{equation}
\text{maximize}\quad \left(\sum_{j=1}^{n}r_j(x_j)\right)\\
\quad \text{to subject} \quad x \geq 0
\quad Ax \leq c^{max}
\end{equation}

اگر به مسئله بالا دقت کنیم میبینم که تابع هدف مسئله ما مقعر است و در اینجا مجموعه نامساوی خطی همان تابع constraint ماست و برای تبدیل به فرم LP تعریف میکنیم تابع زیر را به شکل:

\begin{equation}
\text{min}\quad {\left(p_jx_j,p_jq_j+p^{disc}_j(x_j-q_j)\right)}\\
\end{equation}


که در نتیجه $ r_j $ مقعر است و همچنین میتوانیم بگوییم که 
$ r_j(x_j) \geq u_j $
است اگر و تنها اگر:
\begin{equation}
 p_j(x_j) \geq u_j ,
p^{disc}_j(x_j-q_j) \geq u_j 
\end{equation}

آنگاه با توجه به شرایط بالا میتوانیم فرم LP را تشکیل دهیم:

\begin{center}
$ \text{maximize}\quad \left(\mathbf{1}^Tu\right) $\\
$ \quad \text{to subject} \quad x \geq 0 $\\
$ \quad Ax \leq c^{max}$\\
$\quad p_j(x_j) \geq u_j ,$\\ 
$\quad p^{disc}_j(x_j-q_j) \geq u_j$\\
\end{center}

حال فقط میماند درست کردن x، آخرین مجموعه محدودیت‌ها در LP تضمین می‌کند که 
$ u_i \leq r_i(x) $ ،
که میتوانیم تساوی زیر را جایگزین عبارت پیشین کنیم
\begin{center}
$ ui = ri(x) $
\end{center}
 را بگیریم، در این صورت دو هدف یکسان هستند.


2.2:


\begin{center}
    \includegraphics[width=0.4\textwidth]{2.2.jpg}
\end{center}


1.3:

سوختی که برای بخش i ام مصرف شده برابر است با
$ d_i / s_i \Phi(s_i) $
بنابراین مجموع سوخت استفاده شده برابر است با 
$ \Sigma_{i=1}^{n}d_i / s_i \Phi(s_i) $
وسیله نقلیه در زمان 
$ \tau_i = \Sigma_{j=1}^{i}d_j / s_j $
به نقطه i میرسد بنابراین مسئله بهینه سازی ما به شکل:
\begin{center}
$ \text{minimize}\quad \left(\Sigma_{i=1}^{n}d_i / s_i \Phi(s_i)\right) $\\
$ \quad \text{to subject} \quad s_i^min \leq s_i \leq s_i^max (i = 1,....,n) $\\
$ \tau_i^min \leq \Sigma_{j=1}^{i}d_j / s_j \leq \tau_i^max (i = 1,....,n) $\\
\end{center}
 این مسئله یک مسئله محدب نیست تابغ هدف ما باید در $ s_i $ محدب باشد و همچنین نا برابری ها محدب نیستند با تغییر متغیر میتوانیم این ضعف ها را بپوشانیم. اگر تعریف کنیم 
$ x_i = d_i / s_i $
آنگاه مسئله بهینه سازی ما به شکل:
\begin{center}
$ \text{minimize}\quad \left(\Sigma_{i=1}^{n}x_i \Phi(d_i/x_i)\right) $\\
$ \quad \text{to subject} \quad d_i/s_i^max \leq x_i \leq d_i/s_i^min (i = 1,....,n) $\\
$ \tau_i^min \leq \Sigma_{j=1}^{i}x_j \leq \tau_i^max (i = 1,....,n) $\\
\end{center}
این مسئله محدب است تابع 
$ x_i \Phi(d_i/x_i) $
پرسپکتیوی از $ \Phi $ محدب است بنابراین تابع هدف ما محدب است چون جمع وزن های توابع محدب است و همچنین constraint های ما در x خطی هستند و اگر مسئله را برای 
$ x_i* $
حل کنیم سرعت بهینه را با استفاده از 
$ s_i* = d_i/x_i* $
بدست خواهیم آورد.
2.3:

\begin{center}
    \includegraphics[width=0.9\textwidth]{3.2.jpg}
\end{center}

1.4:

$1/x$
محدب نیست مگر اینکه دامنه را به 
$ R^{++} $ 
محدود کرد در واقع میتوانیم بنویسیم 
$ invpos(x) + invpos(y) $ 
که تابع invpos دامنه اش
$ R^{++} $ 
بنابراین قید های 
$ x \ge 0 , y \ge 0 $ 
را شامل خواهد شد.

2.4:

xy
مقعر نیست ولی میتوانیم قیدمان را به شکل
$  x \geq invpos(y) $
بنویسیم همچنین اگر در اینجا جای x , y را عوض کنیم باز هم همه چیز اوکی خواهد بود.

3.4:

اینجا هم مسئله وقتی مشکل دار خواهد شد که میخواهیم یک تابع محدب را بر مقعر تقسیم کنیم.اگر بنویسیم:
$ quadoverlin(x + y , sqrt(y)) \leq x - y + 5 $
چون تابع quadoverlin در آرگومان دوم یکنواخت کاهش می یابد، بنابراین می تواند یک تابع مقعر را در اینجا بپذیرد، و sqrt مقعر است

4.4:

مشکل اینجاست که xy مقعر نیست، که باعث می شود CVX دستور را رد کند.برای اصلاح این موضوع:
$ \sqrt{xy - z^2} = \sqrt{y(x-z^2/y)} $
حال میتوانیم قید ها را به شکل زیر فرمول بندی کنیم:
$ x + z <= 1 + geomean([x - quadoverlin(z,y), y]) $
این کار می کند، زیرا geomean در هر آرگومان مقعر و بدون کاهش است. بنابراین یک تابع مقعر را در آرگومان اول خود می پذیرد.



\begin{center}
  %  \includegraphics[width=0.9\textwidth]{3.5.jpg}
\end{center}
